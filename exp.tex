\documentclass{amsart}
\usepackage{sagetex}
\usepackage[latin1]{inputenc}
\usepackage{amsmath,amssymb,amsthm}
\usepackage[landscape,letterpaper,top=0.45in,left=0.5in,right=0.5in,bottom=0.4in]{geometry}
\usepackage[american]{babel}
%\usepackage{auto-pst-pdf}
\usepackage[T1]{fontenc}
\usepackage{pst-all}
\usepackage{array,booktabs}
\usepackage{multicol}
\usepackage{cancel}
\usepackage{mathtools}

\usepackage[T1]{fontenc}

\author{Phil Huffman}
\date{\today}

\columnsep=18pt

\setlength{\parindent}{0in}
\usepackage{fancyhdr}
\setlength{\headheight}{15pt}
 
\pagestyle{fancyplain}
 
\lhead{}
\chead{\textbf{\Large{Solving Exponential and Logarithmic Equations}}}
\rhead{}
\lfoot{\tiny{Prepared by Phil Huffman using \LaTeX}}
\cfoot{}
\rfoot{\textcopyright\tiny{\today Jaime Fertig}}

\begin{document}
\bigskip
\begin{footnotesize}
\begin{multicols}{4}
\section*{Exponential Equations}
These equations have a variable in an exponent.
\subsection*{Type I}
The equation can be set in terms of a common base\\
Procedure:
\begin{enumerate}
\item Rewrite in terms of the same base
\item Set exponents on each side equal
\item Solve 
\end{enumerate}
\subsubsection*{Example 1}
\begin{align*}
3^{x}&=27\\
3^{x}&=3^{3}\\
\Aboxed{x&=3}
\end{align*}
\subsubsection*{Example 2}
\begin{align*}
2^{3y+1}&=\sqrt{2}\\
2^{3y+1}&=2^{\frac{1}{2}}\\
3y+1&=\frac{1}{2}\\
3y&=-\frac{1}{2}\\
\Aboxed{y&=-\frac{1}{6}}
\end{align*}
\subsubsection*{Example 3}
\begin{align*}
\frac{1}{125}&=5^{x-5}\\
\frac{1}{5^{3}}&=5^{x-5}\\
5^{-3}&=5^{x-5}\\
-3&=x-5\\
\Aboxed{x&=2}
\end{align*}
\vspace{1cm}

\columnbreak
\subsection*{Type II}
A common base is unavailable. \\
Procedure:
\begin{enumerate}
\item Isolate the exponential.
\item Take the logarithm of each side.
\item Use the properties of logarithms to simplify the equation.
\item Solve
\end{enumerate}
\subsubsection*{Example 1}
\begin{align*}
5^x&=3^{2x-1}\\
\log\left(5^x\right)&=\log\left(3^{2x-1}\right)\\
x\log 5&=\left(2x-1\right)\log 3\\
x\log 5&=2x\log 3-\log 3\\
x\log 5-2x\log 3&=-\log 3\\
x\left(\log 5-2\log 3\right)&=-\log 3\\
\Aboxed{x&=\frac{-\log 3}{\log 5-2\log 3}}\\
x&\approx\sage{round(numerical_approx(-log(3)/(log(5)-2*log(3))),4)}
\end{align*}
\subsubsection*{Example 2}
\begin{align*}
e^{3t}&=8\\
\ln e^{3t}&=\ln 8\\
3t\ln e&=\ln 8\\
3t&=\ln 8\\
\Aboxed{t&=\frac{\ln 8}{3}}\\
t&\approx\sage{round(numerical_approx(log(8)/3),4)}
\end{align*}
\subsubsection*{Example 3}
\begin{align*}
\frac{1440}{1200}&=\frac{1200e^{2k}} {1200}\\
{\frac{6}{5}}&={e^{2k}}\\
\ln{\frac{6}{5}}&=\ln{e^{2k}}\\
\ln{\frac{6}{5}}&=2k\ln{e}\\
\ln{\frac{6}{5}}&=2k\\
\Aboxed{k &= \frac{\ln\frac{6}{5}}{2}} \\
k&\approx\sage{round(numerical_approx(log(6/5)/2),4)}
\end{align*}

\columnbreak
\section*{Logarithmic Equations}
\subsection*{Type I}
Has one or more logarithms on \textbf{ONE} side\\
Procedure:
\begin{enumerate}
\item Use the properties of logarithms to condense to a \textbf{SINGLE} logarithm.
\item Rewrite the equation in exponential form.
\item Solve
\end{enumerate}
\subsubsection*{Example 1}
\begin{align*}
\log_2\left(2x^2+14\right)&=5\\
%2^{\log_2\left(2x^2+14\right)}&=2^5\\
2^5&=2x^2+14\\
32&=2x^2+14\\
0&=2x^2-18\\
0&=2\left(x^2-9\right)\\
\frac {0} {2} &= \left( x^2-9 \right)\\
0 &= \left( x+3 \right) \left( x-3 \right) \\
\Aboxed{x&=\pm 3}
\end{align*}
\subsubsection*{Example 2}
\begin{align*}
\ln\left(x+1\right)&=2+\ln\left(x-1\right)\\
\ln\left(x+1\right)-\ln\left(x-1\right)&=2\\
\ln\left(\frac{x+1}{x-1}\right)&=2\\
e^2&=\frac{x+1}{x-1}\\
\left(x-1\right)e^2&=x+1\\
xe^2-e^2&=x+1\\
xe^2&=x+e^2+1\\
xe^2-x&=e^2+1\\
x\left(e^2-1\right)&=e^2+1\\
\Aboxed{x&=\frac{e^2+1}{e^2-1}}\\
x&\approx\sage{round(numerical_approx((e^2+1)/(e^2-1)),5)}
\end{align*}

\columnbreak
\subsection*{Type II}
Has one or more logarithms on \textbf{EACH} side\\
Procedure:
\begin{enumerate}
\item Use the properties of logarithms to condense to a \textbf{SINGLE} logarithm on each side.
\item Drop the logarithms and set the quantities equal. 
\item Solve
\end{enumerate}
\subsubsection*{Example 1}
\begin{align*}
\log\left(x+1\right)&=2\log\left(x-1\right)\\
\log\left(x+1\right)&=\log\left(x-1\right)^2\\
x+1&=\left(x-1\right)^2\\
x+1&=x^2-2x+1\\
x^2-3x&=0\\
x\left(x-3\right)&=0\\
\Aboxed{x&=3}\quad \cancel{x=0}
\end{align*}
\textbf{Always} check for extraneous solutions!
Recall that the domain of the logarithm function is $\left(0,\infty\right)$. 
Replacing $x$ with $0$ in example 1 to check shows:\\
\begin{align*}
\log1 &= 2\log\left(-1\right)\\
0&=\text{undefined}
\end{align*}
\subsubsection*{Example 2}
\begin{align*}
\log 6-\log\left(5-r\right)&=\log\left(r+2\right)\\
\log\left(\frac{6}{5-r}\right)&=\log\left(r+2\right)\\
\frac{6}{5-r}&=\frac{r+2}{1}\\
\left(5-r\right)\left(r+2\right)&=6\cdot 1\\
5r+10-r^2-2r&=6\\
r^2-3r-4&=0\\
\left(r-4\right)\left(r+1\right)&=0\\
\Aboxed{r&=-1\quad r=4}
\end{align*}
\end{multicols}
\end{footnotesize}
\end{document}
