\documentclass{amsart}
\usepackage[margin=1in,centering]{geometry}
\usepackage{array} % for better arrays (eg matrices) in maths
\usepackage{paralist} % very flexible & customisable lists (eg. enumerate/itemize, etc.)
\usepackage{subfig} % make it possible to include more than one captioned figure/table in a single float
\usepackage{booktabs} % for much better looking tables
\usepackage{verbatim} % adds environment for commenting out blocks of text & for better verbatim
\usepackage{tikz} % adds environment for commenting out blocks of text & for better verbatim
\usepackage{sagetex} % adds environment for commenting out blocks of text & for better verbatim
\usepackage{cmbright}
\usepackage[american]{babel}
% 
\usepackage{fancyhdr}
\pagestyle{fancyplain} 
\fancyhead[co]{\textbf{\huge{Pascal's Triangle}}}
\fancyfoot[l]{\tiny{Prepared using \LaTeX}}
\fancyfoot[r]{\textcopyright\tiny{\today Philip R. Huffman}}
\fancyfoot[c]{}
\linespread{1.5}
\begin{document}
\renewcommand{\headrulewidth}{0pt}
The Binomial Coefficient $\dbinom{m}{n}$ for the expansion of $(a+b)^n$ can be found using Pascal's triangle:
\[1\] \[1\qquad1\] \[1\qquad2\qquad1\] \[1\qquad3\qquad3\qquad1\] \[1\qquad4\qquad6\qquad4\qquad1\] \[1\qquad5\qquad10\qquad10\qquad5\qquad1\]
\[1\qquad6\qquad15\qquad20\qquad15\qquad6\qquad1\] \[1\qquad7\qquad21\qquad35\qquad35\qquad21\qquad7\qquad1\]
\[1\qquad8\qquad28\qquad56\qquad70\qquad56\qquad28\qquad8\qquad1\]
\[1\qquad9\qquad36\qquad84\qquad126\qquad126\qquad84\qquad36\qquad9\qquad1\]
Notice the patterns:
\begin{enumerate}
    \item Each row has one more coefficient than the previous row.
    \item Each coefficient is the sum of the two coefficients above it.
    \item $\dbinom{m}{n}=\dfrac{m!}{n!(m-n)!}$\\
    Where $m$ and $n$ are nonnegative integers and $n\le m$. $m!\text{ (read ``m factorial'') } = 1\times2\times3\ldots m,\; 0!=1$
    \item The $r$th term of an expansion is $\binom{n}{r-1}a^{n-r+1}b^{r-1}$
\end{enumerate}
\subsection*{Example:}Expand $(x+3)^5$.
\[(x+3)^5\]
\[1(x^5)(3^0)+5(x^4)(3^1)+10(x^3)(3^2)+10(x^2)(3^3)+5(x^1)(3^4)+1(x^0)(3^5)\]
Notice that the exponents in each term add up to 5 and that the exponent of $x$ decreases by one in each term.
\[x^5+15x^4+90x^3+270x^2+405x+243\]
\subsection*{Example:}Expand $(2x-y^3)^7$
\[\left[2x+(-y^3)\right] ^7\]
\[(2x)^7+7(2x)^6(-y^3)+21(2x)^5\left(-y^3\right)^2+35(2x)^4\left(-y^3\right) ^3 + 35(2x)^3\left(-y^3\right) ^4 + 21(2x)^2\left(-y^3\right)^5+7(2x)\left(-y^3\right)^6+\left(-y^3\right)^7\]
\[128x^7-448x^6y^3+672x^5y^6-560x^4y^9+280x^3y^{12}-84x^2y^{15}+14xy^{18}-y^{21}\]
\end{document}